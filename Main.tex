% Laget av Joakim Ellestad
% A line starting with % is a comment. In some cases, I have included a command preceded by a %. You may activate the command by removing the %.
%*********************Informasjon om oppsettet*****************************
% Denne \latex template kan brukes for lengre rapporter med flere kapitler, små bøker og avhandlinger. Det gjøres med "report" i `\documentclass[(...)]{report}`, for mindre rapporter kan "article" erstatte report. 

%*******************Informasjon om siteringer*****************************
% Denne templaten bruker Harvard sitering med "agsm" stilen. 
% Alle kidler går inn i kilder.bib. Pass at riktige felter er fylt inn i kilder.bib
% kilde: https://guides.lib.monash.edu/ld.php?content_id=8481587
% Kilde: https://en.wikibooks.org/wiki/LaTeX/Bibliography_Management


%*****************Hvordan skrive*********************************
% Del opp i så mange filer som mulig. Det gjør det enkelt å arbeide i. Eks. hvert kapittel i egen fil. De er nummerert 00,01,02 for sorteringen sin skyld, ingen annet grunn. Hvert kapittel starter med \chapter evt. \chapter*, merk stjerna. Det fjerner "Chapter 1", men lar kapittel tittelen være. Evt. kan man bruker en renewcommand for å erstatte chapter 1 med kapittel 1 f.eks. 

% Videre brukes \section for å dele opp kapittele videre. \subsection kan brukes, men for guds skyld ikke bruk \subsubsection, da har det gått for langt. 
%***************************************************************

%%===========================================
\documentclass[a4paper,12pt]{report}
\usepackage{usepackage}
%%===========================================


%%============================================
\begin{document}
%This is the front page
% For en offisiell NTNU forside kan NTNU  grafisk senter produsere en forside https://ntnu-gs.initio.no/start/ gratis og enkelt å bruke. 
%%=========================================
\thispagestyle{empty}
\begin{figure}
    \centering
    \includegraphics[scale=0.25]{fig/Sample-Logo-square.png}
\end{figure}
\mbox{}\\[5pc]
\begin{center}
\Huge{Project Name}\\[2pc]

\small{Members} \\[3pc]

\large{Spring 2018}\\
Quote goes here\\
\end{center}
\vfill

\noindent \includegraphics[scale=0.2]{fig/NTNULOGO}
 % This is the titlepage
\setcounter{page}{0}
\pagenumbering{arabic}
\renewcommand{\contentsname}{Innhold}
\tableofcontents
\setcounter{page}{0}
\pagenumbering{arabic}
%=============================================
% Bruk include for å legge til flere .tex filer
\setcounter{chapter}{1}
\chapter*{Innledning}
\section{Ny forskning viser at jorda er flat}
Clickbait much...\citet{RN46}

\includepdf[pages=-]{pdf/siteringhjelp.pdf}
\input{inc/02metode.tex}

%%============================================
%%\bibliographystyle{unsrt}
\bibliographystyle{agsm}
%\addcontentsline{toc}{chapter}{\bibname}
\renewcommand\bibname{Kilder}
\bibliography{kilder.bib} 
%%=============================================
\end{document}
